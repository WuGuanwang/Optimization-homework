\documentclass{article}
\usepackage{ctex}
\usepackage{amsmath}
\usepackage{amsthm}
\usepackage{booktabs}
\usepackage{multirow}
\usepackage{geometry}
\usepackage{forest}
\usepackage{amssymb}
\geometry{a4paper, margin=1in}

\title{优化实用算法第三次作业报告}
\author{3230104932 吴官旺}
\date{\today}

\begin{document}
\maketitle
\section*{第一题}
\begin{proof}
    由于$B_k$是对称正定的,那么$B^{-1}_k$也是对称正定的。
    由于$p_k$是方程\[q_k(p) = \frac{1}{2}p^TB_kp+\nabla f(x_k)^Tp +f(x_k)\]的最小值点,则可知
    \[p_k = -(B_k)^{-1}\nabla f(x_k)\]
    由题可知:
    \[f(x_k+p_k)-f(x_k) = \nabla f(x_k)^Tp_k +O(|p_k|) = -\nabla f(x_k)^T(B_k)^{-1}\nabla f(x_k) +O(|p_k|) \]
    由于$B^{-1}_k$是对称正定的,所以$ -\nabla f(x_k)^T(B_k)^{-1}\nabla f(x_k)<0$,即有
    \[f(x_k+p_k)-f(x_k)  < 0\]所以$p_k$是下降方向
\end{proof}

\section*{第二题}
\begin{proof}
    为了方便起见,我们记$g = \nabla f(x_k)$\\
    由于我们是在$g$方向上搜索,且搜索区间不能超过$\Delta_k$,所以我们不妨设$q_k(p)$的最优解为
    \[p^* = t_k \frac{\Delta_k}{\lVert g \rVert}g\]
    其中$t_k \in \left[-1,1\right]$,则有
    \begin{align*}
        q_k(p^*) &= \frac{1}{2}p^{*T}B_kp^* + g^Tp^* +f(x_k)\\
        &= \frac{1}{2}t_k^2\Delta_k^2\frac{g^TB_kg}{\lVert g \rVert^2} + t_k\Delta_k \lVert g \rVert +f(x_k)
    \end{align*}
    记\[ \pi(t) = \frac{1}{2}t^2\Delta_k^2\frac{g^TB_kg}{\lVert g \rVert^2} + t\Delta_k \lVert g \rVert + f(x_k)\]
    这是一个二次函数,且对称轴为$$t^* = -\frac{\lVert g \rVert^3}{\Delta_k g^TB_kg}$$
    \begin{itemize}
        \item 若$g^TB_kg<0$,则二次函数开口向下,且$t^*>0$,所以最优解为$t_k = -1$,即\[p^* = -\frac{\Delta_k}{\lVert g \rVert}g\]
        \item 若$g^TB_kg=0$,则二次函数为一次函数,且斜率大于0,所以最优解为$t_k = -1$,即\[p^* = -\frac{\Delta_k}{\lVert g \rVert}g\]
        \item 若$g^TB_kg>0$,则二次函数开口向上,且$t^*<0$,则有两种情况:
        \begin{itemize}
            \item 若$t^* \geq -1$,则最优解为$t_k =-\frac{\lVert g \rVert^3}{\Delta_k g^TB_kg}$,即\[p^* = -\frac{\lVert g \rVert^2}{g^TB_kg}g\]
            \item 若$t^* \leq -1$,则最优解为$t_k = -1$,即\[p^* = -\frac{\Delta_k}{\lVert g \rVert}g\]
        \end{itemize}
    \end{itemize}
    综上所述,我们得到了$p^*$的表达式,即
    \[p^* = \begin{cases}
        -\frac{\Delta_k}{\lVert g \rVert}g,& g^TB_kg \leq 0\\
        \min\{ -\frac{\lVert g \rVert^3}{\Delta_k g^TB_kg},-1\}\frac{\Delta_k}{\lVert g \rVert}g,& g^TB_kg >0\\
        \end{cases}\]
    这恰好就是Cauchy点的定义。所以证得Cauchy点为$q_k(p)$在$g$方向上的最优解。
\end{proof}

\section*{第三题}
\begin{proof}
    我们只需要证明:\[(B+\lambda I)\sum_{i=1}^{n}\frac{q_i^Tg}{\lambda_i+\lambda}q_i = g\]
    而由题目可知:\[B +\lambda I = Q \Lambda Q^T + \lambda I = Q\begin{bmatrix}
        \lambda_1+\lambda & & & \\
        & \lambda_2+\lambda & &  \\
        & & \ddots& \\
        & & & \lambda_n +\lambda\\ 
    \end{bmatrix}Q^T\]
    \[\sum_{i=1}^{n}\frac{q_i^Tg}{\lambda_i+\lambda}q_i = \sum_{i=1}^{n}\frac{q_iq_i^T}{\lambda_i+\lambda}g =Q\begin{bmatrix}
        \frac{1}{\lambda_1+\lambda} & & & \\
        & \frac{1}{\lambda_2+\lambda} & &  \\
        & & \ddots& \\
        & & & \frac{1}{\lambda_n +\lambda}\\ 
    \end{bmatrix}Q^Tg\] 
    所以有\begin{align*}
        (B+\lambda I)\sum_{i=1}^{n}\frac{q_i^Tg}{\lambda_i+\lambda}q_i = &Q\begin{bmatrix}
        \lambda_1+\lambda & & & \\
        & \lambda_2+\lambda & &  \\
        & & \ddots& \\
        & & & \lambda_n +\lambda\\ 
    \end{bmatrix}Q^TQ\begin{bmatrix}
        \frac{1}{\lambda_1+\lambda} & & & \\
        & \frac{1}{\lambda_2+\lambda} & &  \\
        & & \ddots& \\
        & & & \frac{1}{\lambda_n +\lambda}\\ 
    \end{bmatrix}Q^Tg \\
    =&Q\begin{bmatrix}
        \lambda_1+\lambda & & & \\
        & \lambda_2+\lambda & &  \\
        & & \ddots& \\
        & & & \lambda_n +\lambda\\ 
    \end{bmatrix}\begin{bmatrix}
        \frac{1}{\lambda_1+\lambda} & & & \\
        & \frac{1}{\lambda_2+\lambda} & &  \\
        & & \ddots& \\
        & & & \frac{1}{\lambda_n +\lambda}\\ 
    \end{bmatrix}Q^Tg \\
    =&QQ^Tg \\
    =&g
    \end{align*}
    证毕\\
    令\[\Lambda^* =\begin{bmatrix}
        \frac{1}{\lambda_1+\lambda} & & & \\
        & \frac{1}{\lambda_2+\lambda} & &  \\
        & & \ddots& \\
        & & & \frac{1}{\lambda_n +\lambda}\\ 
    \end{bmatrix} \]
    由题可知:
    \[\lVert p(\lambda) \rVert  = p(\lambda)^T p(\lambda) = g^T Q \Lambda^* Q^T Q \Lambda^* Q^T g = g^T Q \Lambda^{*2} Q^T g \]
    那么\begin{align*}
        \frac{d\lVert p(\lambda) \rVert}{d\lambda} &= g^T Q \frac{d\Lambda^{*2}}{d\lambda} Q^T g \\
        &= g^T Q \begin{bmatrix}
        -\frac{2}{(\lambda_1+\lambda)^3} & & & \\
        & -\frac{2}{(\lambda_2+\lambda)^3} & &  \\
        & & \ddots& \\
        & & & -\frac{2}{(\lambda_n +\lambda)^3}\\ 
    \end{bmatrix} Q^T g \\
        &=-2\sum_{i=1}^{n}\frac{(q_i^Tg)^2}{(\lambda_i+\lambda)^3}
    \end{align*}
\end{proof}

\section*{第四题}
记 \(g:=g_k\),\(B:=B_k\),\(\Delta:=\Delta_k\),\(v:=B^{-1}g\)。则精确解为
\[
p_k=\begin{cases}
-\,B^{-1}g, & \text{若 }\|B^{-1}g\|\le\Delta,\\[4pt]
-\, (B+\lambda^* I)^{-1} g, & \text{其他情况}
\end{cases}
\]
\[\text{其中 }\lambda^*>0\text{ 为唯一满足 }\|(B+\lambda I)^{-1}g\|=\Delta\text{ 的数}\]


\begin{proof}
记 \(S:=\operatorname{span}\{g,v\}\)。函数 \(m\) 在 \( \mathbb{R}^n\) 上严格凸,且无约束极小点为
\[
p_N=-B^{-1}g=-v.
\]
若 \(\|p_N\|\le\Delta\),则 \(p_N\in S\) 且为可行点,故为最优解。

若 \(\|p_N\|>\Delta\),则最优解位于约束集的边界。引入拉格朗日乘子 \(\lambda\ge0\),由 KKT 条件得
\[
\nabla m(p)+\lambda p=0 \quad\Longrightarrow\quad (B+\lambda I)p=-g.
\]
互补松弛和可行性给出 \(\lambda(\Delta-\|p\|)=0\) 且 \(\|p\|\le\Delta\)。因 \(\|p_N\|>\Delta\) 而最优点处约束活跃,故 \(\lambda>0\) 且 \(\|p\|=\Delta\)。于是任意 KKT 解可写为
\[
p(\lambda)=- (B+\lambda I)^{-1} g,\qquad \lambda>0,
\]
且 \(\lambda\) 需满足一标量方程
\[
\phi(\lambda):=\|p(\lambda)\|^2-\Delta^2=0.
\]

证明 \(\phi\) 在 \((0,\infty)\) 上有且仅有一根。对 \(B\) 作正交对角化 \(B=Q\Lambda Q^T\),\(\Lambda=\operatorname{diag}(\mu_i)\),\(\mu_i>0\),令 \(\tilde g=Q^T g\)。则
\[
\|(B+\lambda I)^{-1}g\|^2=\sum_i\frac{\tilde g_i^2}{(\mu_i+\lambda)^2}.
\]
每项关于 \(\lambda\ge0\) 严格递减,故 \(\|(B+\lambda I)^{-1}g\|\) 及 \(\phi(\lambda)\) 在 \([0,\infty)\) 上严格递减。且
\[
\phi(0)=\|B^{-1}g\|^2-\Delta^2>0,\qquad \lim_{\lambda\to\infty}\phi(\lambda)=-\Delta^2<0.
\]
因此由介值定理与严格单调性,存在唯一 \(\lambda^*>0\) 使 \(\phi(\lambda^*)=0\)。因此最优解为
\[
p_k=p(\lambda^*)=- (B+\lambda^* I)^{-1} g.
\]

最后,注意对任意 \(\lambda\ge0\) 有 \((B+\lambda I)^{-1}g\in\operatorname{span}\{g,B^{-1}g\}\),因此所得解确实位于 \(S\) 中。证毕。
\end{proof}

\section*{第五题}
\subsection*{(a)}
部分代码如下:
\begin{verbatim}
def f(x):
    return 0.5 * (1 - 2*x[1] + 2*x[1]**2  - 2*x[1]*x[0]**2 + x[0]**4)

def grad_f(x):
    dfdx0 = -2*x[0]*x[1] + 2*x[0]**3
    dfdx1 = -1 + 2*x[1] - x[0]**2
    return np.array([dfdx0, dfdx1])

def hess_f(x):
    d2fdx00 = -2*x[1] + 6*x[0]**2
    d2fdx01 = -2*x[0]
    d2fdx11 = 2
    return np.array([[d2fdx00, d2fdx01],
                     [d2fdx01, d2fdx11]])

if __name__ == "__main__":
    x_star = np.array([1.0,1.0])
    x_1 = np.array([3.0,2.0])   
    tol = 1e-6
    max_iter = 1000
    print("用Newton方法处理题5中的函数...")
    x,num_iter = newton_method_with_explicit_linesearch(f, grad_f, hess_f, x_1, tol=tol, max_iter=max_iter, save_full_data=True,
                filename="question5_newton_results.txt")
    print(f"最终迭代次数为: {num_iter}")
    print(f"最终解为: {x}")
    draw_gradient_convergence(filename="question5_newton_results.txt",save_plot=True,
                              plot_filename="question5_newton_gradient_convergence.png")
    translate(x_star=x_star,input_filename="question5_newton_results.txt",
              output_filename="question5_newton_residual_results.txt"
              )
    draw_residual_convergence(filename="question5_newton_residual_results.txt",save_plot=True,
                              plot_filename="question5_newton_residual_convergence.png")
    print("="*50)

\end{verbatim}
其中newton法的代码如下:
\begin{verbatim}
    def newton_method_with_explicit_linesearch(f, grad_f, hess_f, x0, tol=1e-5, max_iter=1000, 
                                 save_full_data=False, filename="newton_explicit_linesearch_results.txt"):
    data_dir = Path("data")
    data_dir.mkdir(exist_ok=True)
    filepath = data_dir / filename
    
    x = x0.copy()
    
    with open(filepath, 'w') as file:
        if save_full_data:
            file.write("iteration,gradient_norm,x\n")
        else:
            file.write("iteration,gradient_norm\n")
        
        for i in range(max_iter):
            grad = grad_f(x)
            hess = hess_f(x)
            norm_grad = np.linalg.norm(grad)

            if save_full_data:
                file.write(f"{i},{norm_grad},{x.tolist()}\n")
            else:
                file.write(f"{i},{norm_grad}\n")
            
            if norm_grad < tol:
                return x, i
            
            try:
                direction = np.linalg.solve(hess, -grad)  
            except np.linalg.LinAlgError:
                print("Hessian is singular. Using gradient descent direction.")
                direction = -grad 
            else:
                if np.dot(direction, grad) >= 0:
                    print("Hessian is not positive definite. Using gradient descent direction.")
                    direction = -grad
            
            
            def f_alpha(alpha):
                return f(x + alpha * direction)
            alpha_opt = golden_section_search(f_alpha)
            x = x + alpha_opt * direction
    
    print("最大迭代次数达到,未收敛。")
    return x, max_iter
\end{verbatim}
具体其他代码,详见https://github.com/WuGuanwang/Optimization-homework
\subsection*{(b)}
我们可以计算得到:
\[\nabla^2 f(x) = \begin{bmatrix}
-2x_2 + 6x_1^2 & -2x_1 \\
-2x_1 & 2
\end{bmatrix}\]
\[\nabla r(x) = \begin{bmatrix}
    -2x_1  & 1\\
    0 & -1 \\
\end{bmatrix}\\\]
\[\nabla r(x)^T \nabla r(x) = \begin{bmatrix}
    4x_1^2 & -2x_1 \\
    -2x_1 & 2\\
\end{bmatrix}\]
于是\[\nabla^2 f(x) - \nabla r(x)^T \nabla r(x) = \begin{bmatrix}
    -2x_2 + 6x_1^2 - 4x_1^2 & -2x_1 + 2x_1 \\
    -2x_1 + 2x_1 & 2 - 2\\
\end{bmatrix} = \begin{bmatrix}
    -2x_2 + 2x_1^2 & 0 \\
    0 & 0\\
\end{bmatrix}\]
当$x \to x^* $时
\[\nabla^2 f(x) - \nabla r(x)^T \nabla r(x) \to \begin{bmatrix}
    0 & 0 \\
    0 & 0 \\
\end{bmatrix}\]
所以\[ \nabla^2 f(x) \to \nabla r(x)^T \nabla r(x) \]

\section*{第六题}
\begin{proof}
    由于\(A^TA\)是对称正定矩阵,所以存在正交矩阵\(Q\)与对角阵\(\Lambda\)使得
    \[A^TA = Q^T\Lambda Q\]
    那么则有
    \[(A^TA + \mu I)x = (Q^T\Lambda Q + \mu I )x = Q^T(\Lambda + \mu I)Qx = -A^Tr\]
    解得
    \[x = -Q^T(\Lambda +\mu I)^{-1}QA^Tr\]
    于是有
    \[Ax + r = -AQ^T(\Lambda +\mu I)^{-1}QA^Tr +r\]
    \begin{align*}
        \lVert Ax+r \rVert &= (Ax+r)^T(Ax+r)\\
        &= (-AQ^T(\Lambda +\mu I)^{-1}QA^Tr +r)^T(-AQ^T(\Lambda +\mu I)^{-1}QA^Tr +r)\\
        &= r^TAQ^T(\Lambda +\mu I)^{-1}QA^TAQ^T(\Lambda +\mu I)^{-1}QA^Tr -2r^TAQ^T(\Lambda +\mu I)^{-1}QA^Tr+r^Tr\\
    \end{align*}
    为了方便起见,令\[B(\mu) = AQ^T(\Lambda +\mu I)^{-1}QA^T\]
    则\[ \lVert Ax+r \rVert = r^TB(\mu)B(\mu)r -2r^TB(\mu)r + r^Tr = r^T(B(\mu)-I)^2r\]
    \[\frac{d}{d \mu} \lVert Ax+r \rVert  = 2r^T(B(\mu)-I) \frac{d}{d \mu}B(\mu)r \]
    注意矩阵 \(AA^T\) 与 \(B(\mu)\) 具有相同的特征向量。设
    \[
    AA^T u=\lambda u,
    \]
    则(若 \(A^Tu\neq0\))有 \(A^TA(A^Tu)=\lambda (A^Tu)\),即 \(\lambda\) 也是 \(A^TA\) 的特征值。对该特征向量计算
    \[
    (A^TA+\mu I)^{-1}A^T u=\frac{1}{\lambda+\mu}A^T u,
    \]
    从而
    \[
    B(\mu)u=A(A^TA+\mu I)^{-1}A^T u=\frac{\lambda}{\lambda+\mu}u.
    \]
    因此,若将 \(B(\mu)\) 对角化,存在正交矩阵 \(P\) 使得
    \[
    B(\mu)=P^T\operatorname{diag}\big(b_i(\mu)\big)P,\qquad
    b_i(\mu)=\frac{\lambda_i}{\lambda_i+\mu},
    \]
    其中 \(\{\lambda_i\}\) 为 \(AA^T\) 的特征值
    由此\[B(\mu) - I = P^T(\text{diag}\big(b_i(\mu)\big)-I)P\]
    \[\frac{d}{d \mu}B(\mu) = P^T\operatorname{diag}\big(b_i'(\mu)\big)P\]
    其中 \(b_i'(\mu) = -\frac{\lambda_i}{(\lambda_i+\mu)^2}\)
    所以\(B(\mu)-I\)与\(\frac{d}{d \mu}B(\mu)\)均为负定矩阵,所以\((B(\mu)-I)\frac{d}{d \mu}B(\mu)\)是对称正定矩阵,所以
    \[\frac{d}{d \mu} \lVert Ax+r \rVert = 2r^T(B(\mu)-I) \frac{d}{d \mu}B(\mu)r >0\]
    所以当\(\mu_2>\mu_1\)时,\(\lVert Ax_2+r \rVert>\lVert Ax_1+r \rVert\).
\end{proof}


\end{document}