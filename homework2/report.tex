\documentclass[fontset=none]{ctexart}  

\usepackage{fontspec}
\setCJKmainfont{Noto Serif CJK SC}[
    BoldFont = {Noto Serif CJK SC Bold},
]

\setmainfont{TeX Gyre Termes}
\setsansfont{TeX Gyre Heros}
\setmonofont{TeX Gyre Cursor}
\usepackage{amsmath}
\usepackage{amsthm}
\usepackage{booktabs}
\usepackage{multirow}
\usepackage{geometry}
\usepackage{listings}
\usepackage{xcolor} 
\usepackage{graphicx}   
\usepackage{caption}    
\usepackage{subcaption} 
\usepackage{float}
\usepackage{enumitem}   
\usepackage{amssymb}  

\lstset{
  language=Python,          
  basicstyle=\ttfamily\small,
  keywordstyle=\color{blue}\bfseries, 
  stringstyle=\color{red},    
  commentstyle=\color{green}\itshape, 
  identifierstyle=\color{black}, 
  showstringspaces=false,    
  numbers=left,              
  numberstyle=\tiny\color{gray}, 
  breaklines=true,           
  frame=single,              
  rulecolor=\color{black!30},
  backgroundcolor=\color{gray!10}, 
  tabsize=4,                
  extendedchars=true,        
  escapeinside=``,          
}

\geometry{a4paper, margin=1in}

\title{优化方法第二次作业报告}
\author{3230104932 吴官旺}
\date{\today}

\begin{document}

\maketitle

\section{第一题:用CG算法优化希尔伯特系统}

\subsection{算法}

CG算法实现如下:

\begin{lstlisting}
def conjugate_gradient_method(A, b, x0=None, tol=1e-5, max_iter=None, save_full_data=False, filename="conjugate_gradient_results.txt"):
    data_dir = Path("data")
    data_dir.mkdir(exist_ok=True)
    filepath = data_dir / filename
    n = len(b)
    if x0 is None:
        x0 = np.zeros(n)
    else:
        x = x0.copy()
    r = A `@` x - b  
    p = -r.copy()
    rsold = r `@` r  
    if max_iter is None:
        max_iter = n+1
    with open(filepath, 'w') as file:
        if save_full_data:
            file.write("iteration,residual_norm,x\n")
        else:
            file.write("iteration,residual_norm\n")
        for i in range(max_iter):
            Ap = A `@` p         
            alpha = rsold / (p `@` Ap) 
            x += alpha * p
            r += alpha * Ap
            rsnew = r `@` r 
            if save_full_data:
                file.write(f"{i},{np.sqrt(rsnew)},{x.tolist()}\n")
            else:
                file.write(f"{i},{np.sqrt(rsnew)}\n")
            if np.sqrt(rsnew) < tol:  
                return x, i+1
            beta = rsnew / rsold      
            p = -r + beta * p         
            rsold = rsnew
    return x, max_iter
\end{lstlisting}

\subsection{实验结果}

\begin{figure}[H] 
    \centering 
    \includegraphics[width=0.8\textwidth]{picture/question1_residual_convergence_5.png} 
    \caption{当n=5时的残差收敛图} 
    \label{fig:q1_n5}
\end{figure}

\begin{figure}[H] 
    \centering 
    \includegraphics[width=0.8\textwidth]{picture/question1_residual_convergence_8.png} 
    \caption{当n=8时的残差收敛图} 
    \label{fig:q1_n8}
\end{figure}

\begin{figure}[H] 
    \centering 
    \includegraphics[width=0.8\textwidth]{picture/question1_residual_convergence_12.png} 
    \caption{当n=12时的残差收敛图} 
    \label{fig:q1_n12}
\end{figure}

\begin{figure}[H] 
    \centering 
    \includegraphics[width=0.8\textwidth]{picture/question1_residual_convergence_20.png} 
    \caption{当n=20时的残差收敛图} 
    \label{fig:q1_n20}
\end{figure}

\subsection{实验分析}

可以看出随着n的增大,残差的锯齿效应越明显,说明CG方法随着n增大收敛会变慢,但仍然在稳定收敛,说明CG方法稳定性好。

\section{第二题:预处理共轭梯度法}

\begin{proof}
令$M=LL^T$,其中L为下三角矩阵,令$\hat{x}=L^Tx$,则
\[\hat{A}\hat{x} = \hat{b}, \quad \hat{A}=L^{-1}AL^{-T},\quad \hat{b}=L^{-1}b\]

对$\hat{A}\hat{x} = \hat{b}$应用CG算法得:
\[
\hat{x}_0 = L^T x_0, \quad \hat{r}_0 = \hat{b} - \hat{A} \hat{x}_0, \quad \hat{p}_0 = \hat{r}_0
\]

对于 \( k = 0, 1, 2, \ldots \),执行:

\begin{enumerate}[label=(\arabic*)]
\item $\displaystyle \hat{\alpha}_k = \frac{\hat{r}_k^T \hat{r}_k}{\hat{p}_k^T \hat{A} \hat{p}_k}$
\item $\hat{x}_{k+1} = \hat{x}_k + \hat{\alpha}_k \hat{p}_k$
\item $\hat{r}_{k+1} = \hat{r}_k - \hat{\alpha}_k \hat{A} \hat{p}_k$
\item $\displaystyle \hat{\beta}_k = \frac{\hat{r}_{k+1}^T \hat{r}_{k+1}}{\hat{r}_k^T \hat{r}_k}$
\item $\hat{p}_{k+1} = \hat{r}_{k+1} + \hat{\beta}_k \hat{p}_k$
\end{enumerate}

\textbf{转换回原始变量:}

\[
\hat{x} = L^T x \Rightarrow x = L^{-T} \hat{x}
\]
\[
\hat{r} = L^{-1} r \Rightarrow r = L \hat{r}
\]
\[
p_k = L^{-T} \hat{p}_k \Rightarrow \hat{p}_k = L^T p_k
\]

\begin{align*}
\hat{\alpha}_k &= \frac{\hat{r}_k^T \hat{r}_k}{\hat{p}_k^T \hat{A} \hat{p}_k} \\
&= \frac{(L^{-1} r_k)^T (L^{-1} r_k)}{(L^T p_k)^T (L^{-1} A L^{-T}) (L^T p_k)} \\
&= \frac{r_k^T L^{-T} L^{-1} r_k}{p_k^T L L^{-1} A L^{-T} L^T p_k} \\
&= \frac{r_k^T M^{-1} r_k}{p_k^T A p_k}
\end{align*}

\begin{align*}
\hat{\beta}_k &= \frac{\hat{r}_{k+1}^T \hat{r}_{k+1}}{\hat{r}_k^T \hat{r}_k} \\
&= \frac{(L^{-1} r_{k+1})^T (L^{-1} r_{k+1})}{(L^{-1} r_k)^T (L^{-1} r_k)} \\
&= \frac{r_{k+1}^T M^{-1} r_{k+1}}{r_k^T M^{-1} r_k}
\end{align*}

\[
\hat{p}_{k+1} = \hat{r}_{k+1} + \hat{\beta}_k \hat{p}_k
\]

左乘 \( L^{-T} \):
\[
L^{-T} \hat{p}_{k+1} = L^{-T} \hat{r}_{k+1} + \hat{\beta}_k L^{-T} \hat{p}_k
\]
\[
p_{k+1} = M^{-1} r_{k+1} + \hat{\beta}_k p_k
\]
\end{proof}


\section{第三题证明:Broyden类更新矩阵的奇异性}

% 移除了内部嵌套的proof环境
\noindent \textbf{问题描述}

考虑Broyden类拟牛顿更新公式:
\[
B_{k+1} = B_k - \frac{B_k s_k s_k^T B_k}{s_k^T B_k s_k} + \frac{y_k y_k^T}{y_k^T s_k} + \phi_k (s_k^T B_k s_k) v_k v_k^T
\]
其中
\[
v_k = \left( \frac{y_k}{y_k^T s_k} - \frac{B_k s_k}{s_k^T B_k s_k} \right)
\]
且
\[
\mu_k = \frac{(s_k^T B_k s_k)(y_k^T H_k y_k)}{(s_k^T y_k)^2}, \quad H_k = B_k^{-1}
\]

当 $\phi = \phi_k^c = \frac{1}{1-\mu_k}$ 时,证明 $B_{k+1}$ 是奇异矩阵。

\noindent 
为简化记号,定义:
\[
a = s_k^T B_k s_k, \quad b = s_k^T y_k, \quad d = y_k^T H_k y_k
\]
则
\[
\mu_k = \frac{a d}{b^2}, \quad \phi_k^c = \frac{1}{1-\mu_k} = \frac{b^2}{b^2 - a d}
\]

\noindent 

定义向量:
\[
z = H_k y_k - \frac{d}{b} s_k
\]

\noindent 

将Broyden类更新公式应用于 $H_k y_k$:
\begin{align*}
B_{k+1} H_k y_k &= \left[ B_k - \frac{B_k s_k s_k^T B_k}{a} + \frac{y_k y_k^T}{b} + \phi_k a v_k v_k^T \right] H_k y_k \\
&= B_k H_k y_k - \frac{B_k s_k s_k^T B_k}{a} H_k y_k + \frac{y_k y_k^T}{b} H_k y_k + \phi_k a v_k v_k^T H_k y_k
\end{align*}

逐项计算:

(1) $B_k H_k y_k = y_k$

(2) $\frac{B_k s_k s_k^T B_k}{a} H_k y_k = \frac{B_k s_k}{a} (s_k^T B_k H_k y_k) = \frac{B_k s_k}{a} (s_k^T y_k) = \frac{b}{a} B_k s_k$

(3) $\frac{y_k y_k^T}{b} H_k y_k = \frac{y_k}{b} (y_k^T H_k y_k) = \frac{d}{b} y_k$

(4) 计算 $v_k^T H_k y_k$:
\begin{align*}
v_k^T H_k y_k &= \left( \frac{y_k}{b} - \frac{B_k s_k}{a} \right)^T H_k y_k \\
&= \frac{y_k^T H_k y_k}{b} - \frac{s_k^T B_k H_k y_k}{a} \\
&= \frac{d}{b} - \frac{s_k^T y_k}{a} = \frac{d}{b} - \frac{b}{a}
\end{align*}

因此:
\[
\phi_k a v_k v_k^T H_k y_k = \phi_k a v_k \left( \frac{d}{b} - \frac{b}{a} \right)
\]

代入 $v_k = \frac{y_k}{b} - \frac{B_k s_k}{a}$:
\begin{align*}
B_{k+1} H_k y_k &= y_k - \frac{b}{a} B_k s_k + \frac{d}{b} y_k + \phi_k a \left( \frac{y_k}{b} - \frac{B_k s_k}{a} \right) \left( \frac{d}{b} - \frac{b}{a} \right) \\
&= y_k \left( 1 + \frac{d}{b} + \phi_k \frac{a}{b} \left( \frac{d}{b} - \frac{b}{a} \right) \right) + B_k s_k \left( -\frac{b}{a} - \phi_k \left( \frac{d}{b} - \frac{b}{a} \right) \right)
\end{align*}

简化系数:

$y_k$ 的系数:
\[
1 + \frac{d}{b} + \phi_k \left( \frac{a d}{b^2} - 1 \right) = 1 + \frac{d}{b} - \phi_k + \phi_k \frac{a d}{b^2}
\]

$B_k s_k$ 的系数:
\[
-\frac{b}{a} - \phi_k \left( \frac{d}{b} - \frac{b}{a} \right) = -\frac{b}{a} (1 - \phi_k) - \phi_k \frac{d}{b}
\]

因此:
\begin{equation}
B_{k+1} H_k y_k = y_k \left( 1 + \frac{d}{b} - \phi_k + \phi_k \frac{a d}{b^2} \right) + B_k s_k \left( -\frac{b}{a} (1 - \phi_k) - \phi_k \frac{d}{b} \right)
\label{eq:BHy}
\end{equation}

\noindent 
由拟牛顿条件,$B_{k+1} s_k = y_k$(与 $\phi_k$ 无关),因此:
\begin{align*}
B_{k+1} z &= B_{k+1} \left( H_k y_k - \frac{d}{b} s_k \right) \\
&= B_{k+1} H_k y_k - \frac{d}{b} B_{k+1} s_k \\
&= B_{k+1} H_k y_k - \frac{d}{b} y_k
\end{align*}

代入式 (\ref{eq:BHy}):
\begin{align*}
B_{k+1} z &= y_k \left( 1 + \frac{d}{b} - \phi_k + \phi_k \frac{a d}{b^2} - \frac{d}{b} \right) + B_k s_k \left( -\frac{b}{a} (1 - \phi_k) - \phi_k \frac{d}{b} \right) \\
&= y_k \left( 1 - \phi_k + \phi_k \frac{a d}{b^2} \right) + B_k s_k \left( -\frac{b}{a} (1 - \phi_k) - \phi_k \frac{d}{b} \right)
\label{eq:Bz}
\end{align*}

\noindent 

令 $\phi_k = \phi_k^c = \frac{b^2}{b^2 - a d}$,计算各项系数:

(1) $y_k$ 的系数:
\begin{align*}
1 - \phi_k + \phi_k \frac{a d}{b^2} &= 1 - \phi_k \left( 1 - \frac{a d}{b^2} \right) \\
&= 1 - \frac{b^2}{b^2 - a d} \left( 1 - \frac{a d}{b^2} \right) \\
&= 1 - \frac{b^2}{b^2 - a d} \cdot \frac{b^2 - a d}{b^2} = 0
\end{align*}

(2) $B_k s_k$ 的系数:
\begin{align*}
& -\frac{b}{a} (1 - \phi_k) - \phi_k \frac{d}{b} \\
&= -\frac{b}{a} \left( 1 - \frac{b^2}{b^2 - a d} \right) - \frac{b^2}{b^2 - a d} \cdot \frac{d}{b} \\
&= -\frac{b}{a} \cdot \frac{- a d}{b^2 - a d} - \frac{b d}{b^2 - a d} \\
&= \frac{b d}{b^2 - a d} - \frac{b d}{b^2 - a d} = 0
\end{align*}

因此,当 $\phi_k = \phi_k^c$ 时,$B_{k+1} z = 0$。

由于 $z$ 是非零向量(一般情况下 $H_k y_k$ 与 $s_k$ 线性无关),故 $B_{k+1}$ 是奇异矩阵。

当 $\phi = \phi_k^c = \frac{1}{1-\mu_k}$ 时,Broyden类更新矩阵 $B_{k+1}$ 是奇异矩阵,这由非零向量 $z = H_k y_k - \frac{y_k^T H_k y_k}{y_k^T s_k} s_k$ 满足 $B_{k+1} z = 0$ 所保证。

\section{第四题:用BFGS算法优化Rosenbrock函数与Powell奇异函数}

BFGS算法如下:

\begin{lstlisting}
def BFGS_method(f, grad_f, x0, tol=1e-6, max_iter=1000, save_full_data=False, 
         filename="BFGS_results.txt"):
    data_dir = Path("data")
    data_dir.mkdir(exist_ok=True)
    filepath = data_dir / filename
    
    n = len(x0)
    x = x0.copy()
    H = np.eye(n)  
    
    with open(filepath, 'w') as file:
        if save_full_data:
            file.write("iteration,gradient_norm,x\n")
        else:
            file.write("iteration,gradient_norm\n")
        
        for i in range(max_iter):
            grad = grad_f(x)
            norm_grad = np.linalg.norm(grad)

            if save_full_data:
                file.write(f"{i},{norm_grad},{x.tolist()}\n")
            else:
                file.write(f"{i},{norm_grad}\n")
            
            if norm_grad < tol:
                return x, i
            
            direction = -H `@` grad 
            
            def f_alpha(alpha):
                return f(x + alpha * direction)
            
            def f_deriv_alpha(alpha):
                return np.dot(grad_f(x + alpha * direction), direction)
            
            alpha_opt = wolfe_search(f_alpha, f_deriv_alpha)
            s = alpha_opt * direction  
            x_new = x + s
            y = grad_f(x_new) - grad  
            
            rho = 1.0 / (y `@` s)  
            I = np.eye(n)
            H = (I - rho * np.outer(s, y)) `@` H `@` (I - rho * np.outer(y, s)) + rho * np.outer(s, s)
            
            x = x_new
    
    print("最大迭代次数达到,未收敛。")
    return x, max_iter
\end{lstlisting}

\subsection{实验结果}

\begin{figure}[H]
  \centering
  \begin{subfigure}[b]{0.45\textwidth} 
    \includegraphics[width=\textwidth]{picture/Rosenbrock_gradient_convergence_n6.png}
    \caption{n=6时的梯度下降图}
    \label{fig:rosen_n6_grad}
  \end{subfigure}
  \hfill 
  \begin{subfigure}[b]{0.45\textwidth}
    \includegraphics[width=\textwidth]{picture/Rosenbrock_residual_convergence_n6.png}
    \caption{n=6时的残差下降图}
    \label{fig:rosen_n6_res}
  \end{subfigure}
  \caption{n=6时Rosenbrock函数实验} 
  \label{fig:rosen_n6}
\end{figure}

\begin{figure}[H]
  \centering
  \begin{subfigure}[b]{0.45\textwidth} 
    \includegraphics[width=\textwidth]{picture/Rosenbrock_gradient_convergence_n8.png}
    \caption{n=8时的梯度下降图}
    \label{fig:rosen_n8_grad}
  \end{subfigure}
  \hfill 
  \begin{subfigure}[b]{0.45\textwidth}
    \includegraphics[width=\textwidth]{picture/Rosenbrock_residual_convergence_n8.png}
    \caption{n=8时的残差下降图}
    \label{fig:rosen_n8_res}
  \end{subfigure}
  \caption{n=8时Rosenbrock函数实验} 
  \label{fig:rosen_n8}
\end{figure}

\begin{figure}[H]
  \centering
  \begin{subfigure}[b]{0.45\textwidth} 
    \includegraphics[width=\textwidth]{picture/Rosenbrock_gradient_convergence_n10.png}
    \caption{n=10时的梯度下降图}
    \label{fig:rosen_n10_grad}
  \end{subfigure}
  \hfill 
  \begin{subfigure}[b]{0.45\textwidth}
    \includegraphics[width=\textwidth]{picture/Rosenbrock_residual_convergence_n10.png}
    \caption{n=10时的残差下降图}
    \label{fig:rosen_n10_res}
  \end{subfigure}
  \caption{n=10时Rosenbrock函数实验} 
  \label{fig:rosen_n10}
\end{figure}

\begin{figure}[H]
  \centering
  \begin{subfigure}[b]{0.45\textwidth} 
    \includegraphics[width=\textwidth]{picture/Powell_singular_gradient_convergence.png}
    \caption{Powell函数梯度下降图}
    \label{fig:powell_grad}
  \end{subfigure}
  \hfill 
  \begin{subfigure}[b]{0.45\textwidth}
    \includegraphics[width=\textwidth]{picture/Powell_singular_residual_convergence.png}
    \caption{Powell函数残差下降图}
    \label{fig:powell_res}
  \end{subfigure}
  \caption{Powell奇异函数实验} 
  \label{fig:powell}
\end{figure}

\subsection{实验分析}

由图可以看出,在使用BFGS方法时,总是先经历一段平缓期,后在接近最优解时迅速下降,这也是拟牛顿方法的特点。

其次可以看出BFGS方法收敛快,并且稳定。

\end{document}