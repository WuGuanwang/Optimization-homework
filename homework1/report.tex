\documentclass{article}
\usepackage{ctex}
\usepackage{amsmath}
\usepackage{amsthm}
\usepackage{booktabs}
\usepackage{multirow}
\usepackage{geometry}

\geometry{a4paper, margin=1in}

\title{优化方法第一次作业报告}
\author{3230104932 吴官旺}
\date{\today}

\begin{document}
\maketitle
\section*{第一题}
\begin{proof}
考虑矩阵范数的定义
\[ \Vert A \Vert = \max_{x \neq 0} \frac{\Vert Ax \Vert }{\Vert x \Vert } \]
则有$\forall x \neq 0$
\[ \Vert A \Vert \geq \frac{\Vert Ax \Vert }{\Vert x \Vert } \]
\[ \Vert Ax \Vert \leq \Vert A \Vert \Vert x \Vert \]
当$x=0$时,显然也成立\\
因此$\Vert Ax \Vert \leq \Vert A \Vert \Vert x \Vert$对于任意$x$与任意$A$均成立。\\
那么则有
\[ \Vert x \Vert = \Vert B^{-1}Bx \Vert \leq \Vert B^{-1} \Vert \Vert Bx \Vert \]
即为
\[ \Vert Bx \Vert \geq \frac{\Vert x \Vert }{\Vert B^{-1} \Vert } \]
\end{proof}

\section*{第二题}
\subsection*{(a)}
\begin{proof}
要验证$(A+ab^T)^{-1}=A^{-1}-\frac{A^{-1}ab^TA^{-1}}{1+b^TA^{-1}a}$,只需验证
\[(A+ab^T)(A^{-1}-\frac{A^{-1}ab^TA^{-1}}{1+b^TA^{-1}a})=I\]
\begin{align*}
&(A+ab^T)(A^{-1}-\frac{A^{-1}ab^TA^{-1}}{1+b^TA^{-1}a})\\
=&AA^{-1}-\frac{AA^{-1}ab^TA^{-1}}{1+b^TA^{-1}a}+ab^TA^{-1}-\frac{ab^TA^{-1}ab^TA^{-1}}{1+b^TA^{-1}a}\\
=&I-\frac{ab^TA^{-1}}{1+b^TA^{-1}a}+ab^TA^{-1}-\frac{ab^TA^{-1}ab^TA^{-1}}{1+b^TA^{-1}a}\\
=&I+ab^TA^{-1}(1-\frac{1+b^TA^{-1}a}{1+b^TA^{-1}a})\\
=&I
\end{align*}
\end{proof}
\subsection*{(b)}
\begin{proof}
已知$H_k^{-1}=B_k$,且$B_k$与$H_k$均为对称矩阵\\
若令$u=y_k-B_ks_k,   v^T=\frac{(y_k-B_ks_k)^{T}}{(y_k-B_ks_k)^Ts_k}$,则$B_{k+1}=B_k+uv^T$\\
由(a)中公式可得
\begin{align*}
&B_{k+1}^{-1}\\
=&B_k^{-1}-\frac{B_k^{-1}u v^T B_k^{-1}}{1+v^T B_k^{-1} u}\\
=&B_k^{-1}-\frac{B_k^{-1}(y_k-B_ks_k) \frac{(y_k-B_ks_k)^{T}}{(y_k-B_ks_k)^Ts_k} B_k^{-1}}{1+\frac{(y_k-B_ks_k)^{T}}{(y_k-B_ks_k)^Ts_k} B_k^{-1}(y_k-B_ks_k)}\\
=&B_k^{-1}-\frac{(B_k^{-1}y_k-s_k)(B_k^{-1}y_k-s_k)^T}{(y_k-B_ks_k)^Ts_k+(y_k-B_ks_k)^{T}B_k^{-1} (y_k-B_ks_k)}\\
=&B_k^{-1}-\frac{(B_k^{-1}y_k-s_k)(B_k^{-1}y_k-s_k)^T}{y_k^T B_k^{-1} y_k-s_k^T B_k^TB_k^{-1} y_k} \\
=&H_k-\frac{(s_k-H_ky_k)(s_k-H_ky_k)^T}{(s_k-H_ky_k)^T y_k}\\
=&H_{k+1}\\
\end{align*}
\end{proof}



\section*{第三题}
问题描述:使用最速下降法和纯牛顿法分别求解Rosenbrock函数和Beale函数的极小值点,比较两种方法的收敛速度和效果。
\subsection*{方法描述}
最速下降法是一种简单的优化算法,通过沿着负梯度方向更新变量来寻找函数的最小值。该方法的优点是实现简单,但在处理复杂函数时可能收敛较慢。

纯牛顿法是一种二阶优化算法,通过利用Hessian矩阵的信息来加速收敛。该方法在接近最优解时表现良好,但计算Hessian矩阵的代价较高。

在本次实验中,我们将这两种方法应用于Rosenbrock函数和Beale函数,并比较它们的收敛速度和效果。
\subsection*{实验结果}
实验代码见附录中的question3.py文件。以下是实验结果:
\begin{itemize}
    \item Rosenbrock函数
    \begin{itemize}
        \item 最速下降法:迭代次数为8753,最终梯度范数为9.92e-06,目标函数值为0.000000。
        \item 纯牛顿法:迭代次数为5,最终梯度范数为8.61e-06,目标函数值为0.000000。
    \end{itemize}
    \item Beale函数
    \begin{itemize}
        \item 最速下降法:迭代次数为10000,最终梯度范数为8.22e-02,目标函数值为0.669522。
        \item 纯牛顿法:迭代次数为40,最终梯度范数为1.97e-17,目标函数值为14.203125。
    \end{itemize}
\end{itemize}
\subsection*{结果分析}
从实验结果可以看出,纯牛顿法在收敛速度和效果上均优于最速下降法。这是因为纯牛顿法利用了Hessian矩阵的信息,使得在接近最优解时能够更快地收敛。然而,最速下降法由于其简单性,在某些情况下仍然具有一定的优势,特别是在计算资源有限的情况下。

\section*{第四题}
问题描述:优化一个四次函数,并观察在在不同初始点和不同$\rho$值下的优化效果。并且对比牛顿法与带线搜索牛顿法的收敛速度与鲁棒性
\subsection*{方法描述}
带线搜索牛顿法是在每次迭代中引入线搜索的策略,以确定最优步长,从而提高收敛速度和鲁棒性。
\subsection*{实验结果}
\subsubsection*{初始点 $x^{(0)} = [\cos(70^\circ), \sin(70^\circ), \cos(70^\circ), \sin(70^\circ)]^T$}

\begin{table}[h!]
\centering
\caption{不同 $\sigma$ 值下的牛顿法性能比较 ($x^{(0)} = [\cos(70^\circ), \sin(70^\circ), \cos(70^\circ), \sin(70^\circ)]^T$)}
\begin{tabular}{ccccc}
\toprule
$\sigma$ 值 & 方法 & 迭代次数 & 最终梯度范数 & 目标函数值 \\
\midrule
\multirow{2}{*}{1} 
& 纯牛顿法 & 8 & $4.85 \times 10^{-6}$ & $0.000000$ \\
& 带线搜索牛顿法 & 5 & $7.13 \times 10^{-7}$ & $0.000000$ \\
\addlinespace
\multirow{2}{*}{$10^4$} 
& 纯牛顿法 & 19 & $1.19 \times 10^{-6}$ & $0.000000$ \\
& 带线搜索牛顿法 & 8 & $6.36 \times 10^{-6}$ & $0.000000$ \\
\bottomrule
\end{tabular}
\end{table}

\subsubsection*{初始点 $x^{(0)} = [\cos(50^\circ), \sin(50^\circ), \cos(50^\circ), \sin(50^\circ)]^T$}

\begin{table}[h!]
\centering
\caption{不同 $\sigma$ 值下的牛顿法性能比较 ($x^{(0)} = [\cos(50^\circ), \sin(50^\circ), \cos(50^\circ), \sin(50^\circ)]^T$)}
\begin{tabular}{ccccc}
\toprule
$\sigma$ 值 & 方法 & 迭代次数 & 最终梯度范数 & 目标函数值 \\
\midrule
\multirow{2}{*}{1} 
& 纯牛顿法 & 8 & $3.11 \times 10^{-6}$ & $0.000000$ \\
& 带线搜索牛顿法 & 5 & $8.69 \times 10^{-7}$ & $0.000000$ \\
\addlinespace
\multirow{2}{*}{$10^4$} 
& 纯牛顿法 & 19 & $8.71 \times 10^{-7}$ & $0.000000$ \\
& 带线搜索牛顿法 & 8 & $7.77 \times 10^{-6}$ & $0.000000$ \\
\bottomrule
\end{tabular}
\end{table}

\subsection*{结果分析}
线搜索的效果:无论$ \sigma$的值是多少,带线搜索的牛顿法收敛更快,但当 $\sigma$ 越大时,两种方法都需要更多迭代次数,而不同初始点对收敛行为有一定影响,但整体模式相似



\section*{第五题}
问题描述:修改版LASSO问题,使用Huber函数平滑了L1范数,使其处处可微。参数 $\mu$ 控制正则化强度,$\delta$ 是平滑参数。
对比最速梯度下降与bb方法的收敛速度差异。
\subsection*{方法描述}
Barzilai-Borwein方法: 一种拟牛顿法,通过一种简单的公式来近似Hessian矩阵,从而产生一个高效的步长选择策略
\subsection*{实验结果}
\subsection*{参数设置: $\mu = 0.01$, $\delta = 10^{-5}$}

\begin{table}[h!]
\centering
\caption{修改版LASSO问题的优化方法比较 ($\mu = 0.01$, $\delta = 10^{-5}$)}
\begin{tabular}{lcccccc}
\toprule
方法 & 收敛状态 & 迭代次数 & 最终梯度范数 & 目标函数值 & 解的稀疏度 & 与真实解的误差 \\
\midrule
最速下降法 & 未收敛 & 50000 & $1.25 \times 10^{-1}$ & 0.594100 & 631/1024 & 1.520776 \\
Barzilai-Borwein 方法 & 收敛 & 8540 & $9.67 \times 10^{-4}$ & 0.507155 & 102/1024 & 0.000517 \\
\bottomrule
\end{tabular}
\end{table}

\subsection*{参数设置: $\mu = 0.001$, $\delta = 10^{-6}$}

\begin{table}[h!]
\centering
\caption{修改版LASSO问题的优化方法比较 ($\mu = 0.001$, $\delta = 10^{-6}$)}
\begin{tabular}{lcccccc}
\toprule
方法 & 收敛状态 & 迭代次数 & 最终梯度范数 & 目标函数值 & 解的稀疏度 & 与真实解的误差 \\
\midrule
最速下降法 & 未收敛 & 50000 & $3.35 \times 10^{-2}$ & 0.083561 & 897/1024 & 3.623616 \\
Barzilai-Borwein 方法 & 未收敛 & 50000 & $6.35 \times 10^{-3}$ & 0.057715 & 603/1024 & 1.261220 \\
\bottomrule
\end{tabular}
\end{table}
\subsection*{结果分析}
BB方法的优势:
在 $\mu=0.01$ 时,BB方法成功收敛,而最速下降法未能收敛,BB方法的收敛速度明显更快,BB方法得到的解更稀疏,说明正则化效果更好

当 $\mu$ 从 0.01 减小到 0.001 时,问题变得更难优化,在 $\mu=0.001$ 时,两种方法都未能收敛到要求的精度

\end{document}